\chapter{Lessons Learned: summarize what you learned individually and as a team}
\label{cp:Lessons Learned}
This chapter is used to summarize what was learned as a group and on an individual level.

\section{Group Learnings}
\label{sec:Lessons Learned}
A significant takeaway was the importance of early adoption and thoroughly reading the documentation relevant to the tasks. While the individual segments of the work were well-structured and effective, achieving seamless interactivity between systems required a more comprehensive understanding of the bigger picture.

This project was an important experience for our ability to work as a team, especially when working with members from diverse specializations. However, a different approach to hardware selection could have improved the process. Instead of starting with minimal hardware and scaling up, we should have opted for hardware with surplus resources, ensuring it could handle the demands of the project. From there, we could have determined the minimal specifications required.

Additionally, assuming that a given technology or framework would work as expected proved to be a flawed approach. This experience taught us to anticipate the need to switch technologies, especially when using unfamiliar tools, and to plan our timeline accordingly. By adopting this mindset, we can better mitigate risks and adapt to challenges in future projects.

\section{Individual Learning: Christoph Schmid}
\label{sec:Individual Learning: Christoph Schmied}
One key takeaway was the need to allocate significantly more time for tasks than initially estimated, as unexpected challenges often arise. Proper time management and realistic planning will be essential in future projects.

Another critical learning was the importance of verifying server specifications in advance. Relying on unverified assumptions about hardware capabilities can lead to delays and inefficiencies. Ensuring that the necessary server configurations are in place before deployment is now a priority for me.

Additionally, I recognized the importance of configuring firewalls appropriately. For example, ensuring that port 22 is open for secure access is a small yet crucial detail that can prevent unnecessary troubleshooting and delays.

When working with PCBs, I learned to multiply the estimated time required by a factor of five. This accounts for the intricate processes involved in design, manufacturing, and testing, which often take longer than expected.

Lastly, I developed a deeper understanding of the importance of encryption. Security considerations should not be an afterthought but an integral part of the planning and implementation process. Ensuring data protection and secure communication will be a priority in my future work.

These lessons have not only enhanced my technical skills but also shaped my approach to project management and problem-solving in a professional setting.

\section{Individual Learning: Romeo Massoud}
\label{sec:Individual Learning: }
Working with React Native to display gas values taught me several things. I gained a deeper understanding of React Native's component-based architecture, state management, and props handling. I also learned how to fetch data from APIs, manage errors, and display real-time updates.

Designing the user interface enhanced my skills in creating responsive layouts and dynamic visualizations, while testing and debugging improved my ability to handle edge cases and ensure app reliability. 

This experience enhanced my technical expertise while also strengthening my problem-solving abilities, teamwork, and time management skills, making it an invaluable learning experience.

\section{Individual Learning: Maceo Pohl}
\label{sec:Individual Learning: Maceo Pohl}
One key learning experience was the importance of carefully analyzing and understanding data sheets. By doing so, I was able to fully grasp the technical specifications and helped me to better utilize the capabilities of the Sensors.

Another significant aspect was learning to collaborate effectively within a group that consisted of individuals with diverse areas of expertise, compared to previous projects where team members typically had similar knowledge levels. Working in a multidisciplinary team allowed me to get a  better understanding about group dynamics and making sure that the different parts of the project will work together.

Additionally, I had the opportunity to work extensively with the ESP32 microcontroller. This hands-on experience allowed me to deepen my understanding of the capabilities and challenges of LoRa and integrating wireless communication technology within a bigger IoT stack. 

%\begin{importantbox}
%\end{importantbox}

%Chris
%- freeRtos ging nicht (fälchliche annahme das es funktioniert)
%- vor implementierung ausprobieren
%- Microcontroler besser auswählen
% PCB zeit x5 nehmen
%- LoRa Modul vorher genutzt haben/ bessere Recherche in die techniken
%- Verschlüsselung gedanken machen
%- generell mehr zeit einplanen
%- Server Specs vorher verifizieren
%- Firewall : port 22 freigeben
%- fgit von anfang an ordentlich nutzen
%- Trello von anfang an richtig nutzen

%Romeo
%- Brich AB!!!
%- Informieren vorab, was es alles gibt und alles einmal anwenden (hello world scenario)
%- Webserver integration vorher testen buildfolder hat nicht funktioniert
