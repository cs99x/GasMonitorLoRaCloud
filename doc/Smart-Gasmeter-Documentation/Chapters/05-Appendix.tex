\chapter{Appendix}

\section{Sprint History}
\subsection{ConOps MS Report - 24. Okt 2024}{
1. Introduction
\newline
This document provides developers and team leaders with an overview of the target users
for the project. It offers a summary of the various use cases and target groups.
\newline
2. System Overview
\newline
Vision: The system is designed to detect multiple harmful gases in real-time and wirelessly
transmit the measured values to a digital device, even without direct line of sight.
Operating Environment: The system will be used both indoors and outdoors by emergency
response units such as public safety agencies (BOS). It provides an additional layer of safety
for responders during operations and supports leadership in monitoring the condition of
personnel and potential hazards at the scene. The operator will carry a mobile gas detection
device that provides information about the gases in their environment. Simultaneously, the
system sends the collected gas concentrations to the cloud, allowing leadership to monitor
the operator’s situation remotely and automatically.
\newline
3. Operational Requirements
Key Functions:
\begin{itemize}
    \item Real-time detection of gases
    \item Display of gas concentrations for the device user
    \item Transmission of gas concentrations to the cloud
    \item Real-time display of the measured gas concentrations via a web app
\end{itemize}
User Groups:
\begin{itemize}
    \item Technical Relief Organization (THW)
    \item Fire Department
    \item Researchers
\end{itemize}
The system allows users to determine whether hazardous gases are present during the
exploration of scenarios, operational situations, or terrains. At the same time, command
personnel or those outside the scenario can monitor whether the user is in a dangerous
situation and whether assistance or rescue is required. It also helps to create a clearer
picture of the situation by identifying hazardous zones.
Main Application Area: The system will primarily be used by the Technical Relief
Organization (THW). It enables emergency personnel to detect dangerous gases in their
environment and allows command personnel to monitor the status of the team in real time.
Operational Requirements:
\begin{itemize}
    \item Use in confined spaces with high dust levels
    \item Shock resistance
    \item Resistance to sparks
    \item Long-term usability, even during hours-long standby operations and explorations
\end{itemize}
spanning several kilometers
Potential Areas of Use:
\begin{itemize}
    \item Forests
    \item Bunker facilities
    \item Industrial areas (chemical plants, landfills, farms)
    \item Residential complexes
    \item Rubble zones
    \item Flood zones
\end{itemize}

Cave Exploration: Scientists working in caves or other enclosed spaces can better assess
their surroundings. External observers can also monitor the situation and intervene more
quickly if needed. Additionally, the collected data can be used for later analysis or general risk
assessment.
Private Users in Health-Risk Areas: Individuals in hazardous areas such as abandoned
buildings (Lost Places) can better assess their own safety. External observers can also
monitor the situation and request assistance early in case of danger, as well as relay
important information
}

\subsection{Architectural Spike MS Report -  21. Nov 2024}{
Progress Since the Last Milestone:
\begin{itemize}
    \item Successfully reassigned workload across the team.
    \item eveloped a "Hello World" iOS mobile application.
    \item  Created the first prototype of a React Native website
    \item Developed the initial prototype of a React Native mobile application.
    \item Built a "Hello World" project using Django.
    \item Nearly completed the setup of the BWShare Cloud Server.
\end{itemize}
Plans for the Next Milestone:
\begin{itemize}
    \item Deliver the first full-flow prototype, integrating key components.
\end{itemize}
Challenges Faced and Resolutions:
\begin{itemize}
    \item Team Reshuffling
    \item Challenge: Adjusting team roles and responsibilities caused temporary
delays.
    \item Resolution: Successfully restructured the team to ensure alignment with
project goals.
    \item Setting Up the Server for the Django Backend:
    \item Challenge: Encountered technical issues during the configuration of the
backend server.
    \item Resolution: Continued efforts to resolve these issues, with targeted support
    \item Scaling for Gas Metrics:
    \item Challenge: Implementation of scalable systems for gas measurement
presented difficulties.
    \item Resolution: work in progress
\end{itemize}
}

\subsection{Alpha MS Report - 19. Dez 2024}

{What did we accomplish since the last milestone?}
\begin{itemize}
    \item Read sensor data successfully.
    \item Tested the sensor with limited capacity.
    \item Prepared different data interfaces for user testing.
    \item Identified various hardware components to resolve the battery management issue.
\end{itemize}

\subsection*{What do we plan to do by the next milestone?}
\begin{itemize}
    \item Integrate all necessary components to work seamlessly together.
    \item Implement non-blocking code for the handheld device.
    \item Ensure all four sensors work simultaneously.
    \item Collect user feedback on the app design.
    \item Develop initial designs for the cases of the different hardware components.
\end{itemize}

\subsection*{What hindrances did we face?}
\begin{itemize}
    \item Lack of necessary hardware (e.g., cables).
    \begin{itemize}
        \item \textbf{Solution:} Acquire the required hardware.
    \end{itemize}
    \item Uncertainty about the interface design.
    \begin{itemize}
        \item \textbf{Solution:} Conduct user testing.
    \end{itemize}
    \item Unknown limits of the sensors (e.g., sensitivity and accuracy).
    \begin{itemize}
        \item \textbf{Solution:} Engage a chemist to test the sensors.
    \end{itemize}
\end{itemize}


\subsection{Beta MS Report}

\subsection*{Accomplishments Since the Last Milestone}
\begin{enumerate}
    \item Achieved successful integration of all components.
    \item Display functionality implemented:
    \begin{itemize}
        \item Shows all gases.
        \item Cycles through cases with a button.
    \end{itemize}
\end{enumerate}

\subsection*{Plans for the Next Milestone}
\begin{enumerate}
    \item Design and finalize the housing for the handheld device and gateway.
    \item Enhance the display and button functionality.
    \item Improve general documentation.
    \item Upgrade the gateway app.
\end{enumerate}

\subsection*{Challenges and Solutions}
\begin{enumerate}
    \item Display Driver Blocking LoRa Communication:
    \begin{itemize}
        \item \textbf{Solution:} Adjusted the communication process to allocate more time (resolved blocking code issue).
    \end{itemize}
    \item Bad Sensor Calibration:
    \begin{itemize}
        \item \textbf{Solution:} Focus on improving calibration methods.
    \end{itemize}
\end{enumerate}

\section{Architectural Decision Records}
\begin{itemize}
    \item Changed from ESP32 S1 to ESP32 S3 (more memory needed).
    \item Reduced the number of bars on the screen where all sensors are shown (conflict with LoRa transmission).
    \item Switched from React Native to HTML (unable to create a build folder for integration into the Flask server).
\end{itemize}
\newpage
\section{Code Report}
This section presents a breakdown of the project's codebase using \texttt{cloc}.


\begin{listing}[H]
\begin{minted}[fontsize=\small, frame=lines]{text}
34 text files.
      24 unique files.
      17 files ignored.

github.com/AlDanial/cloc v 2.02  T=0.02 s (1460.0 files/s, 144238.9 lines/s)
\end{minted}
\caption{cloc Analysis Summary}
\end{listing}

\section{Lines of Code Breakdown}

\begin{table}[H]
\centering
\caption{Lines of Code Breakdown by Language}
\begin{tabularx}{\textwidth}{@{}lrrr@{}}
\toprule
\textbf{Language}       & \textbf{Files} & \textbf{Blank Lines} & \textbf{Code Lines} \\ \midrule
Markdown                & 9              & 209                  & 586                 \\
HTML                    & 2              & 39                   & 426                 \\
Arduino Sketch          & 2              & 73                   & 310                 \\
Swift                   & 4              & 36                   & 196                 \\
Python                  & 1              & 33                   & 158                 \\
JSON                    & 3              & 0                    & 52                  \\
Text                    & 1              & 0                    & 28                  \\
XML                     & 2              & 0                    & 21                  \\ \midrule
\textbf{SUM:}           & 24             & 390                  & 1777                \\ \bottomrule
\end{tabularx}
\end{table}

\section{Authors Report}

\begin{tcolorbox}[colback=red!5!white, colframe=red!75!black, title=Note]
The authors listed in the repository may differ from those in the Git commits because most files were uploaded from a single PC to maintain consistency within the Git repository.
\end{tcolorbox}

\subsection{full file listing}{
\begin{itemize}
    \item \begin{verbatim} README.md (Romeo Massoud) \end{verbatim}
    \item \begin{verbatim} graph.html (Romeo Massoud) \end{verbatim}
\item \begin{verbatim} graph.html (Romeo Massoud) \end{verbatim}
\item \begin{verbatim} index.html (Romeo Massoud) \end{verbatim}
\item \begin{verbatim} __init__.py (Romeo Massoud, Christoph Schmid) \end{verbatim}
\item \begin{verbatim} app.py (Romeo Massoud, Christoph Schmid) \end{verbatim}
\item \begin{verbatim} sensor_data.db (Auto/untouched) \end{verbatim}
\item \begin{verbatim} README.md (Romeo Massoud, Christoph Schmid) \end{verbatim}
\item \begin{verbatim} requirements.txt (Romeo Massoud, Christoph Schmid) \end{verbatim}
\item \begin{verbatim} Smart gasmeter documentation (Maceo Pohl, Romeo Massoud, Christoph Schmid) \end{verbatim}
\item \begin{verbatim} README.md (Maceo Pohl) \end{verbatim}
\item \begin{verbatim} README.md (Maceo Pohl) \end{verbatim}
\item \begin{verbatim} handheld-gasmeter.ino (Maceo Pohl) \end{verbatim}
\item \begin{verbatim} README.md (Maceo Pohl) \end{verbatim}
\item \begin{verbatim} lora-bluetooth-gateway.ino (Christoph Schmid, Maceo Pohl) \end{verbatim}
\item \begin{verbatim} gasmeter_handheld_device.kicad_ (Christoph Schmid) \end{verbatim}
\item \begin{verbatim} README.md (Christoph Schmid) \end{verbatim}
\item \begin{verbatim} gasmeter_handheld_device.pdf (Auto/untouched) \end{verbatim}
\item \begin{verbatim} gasmeter_handheld_device.png (Auto/untouched) \end{verbatim}
\item \begin{verbatim} gasmeter_handheld_device.kicad_pcb (Christoph Schmid) \end{verbatim}
\item \begin{verbatim} gasmeter_handheld_device.kicad_pro (Christoph Schmid) \end{verbatim}
\item \begin{verbatim} gasmeter_handheld_device.kicad_sch (Christoph Schmid) \end{verbatim}
\item \begin{verbatim} gasmeter_handheld_device.kicad_pcb (Christoph Schmid) \end{verbatim}
\item \begin{verbatim} gasmeter_handheld_device.kicad_pro (Christoph Schmid) \end{verbatim}
\item \begin{verbatim} gasmeter_handheld_device.kicad_sch (Christoph Schmid) \end{verbatim}
\item \begin{verbatim} README.md (Christoph Schmid) \end{verbatim}
\item \begin{verbatim} contents.xcworkspacedata (Auto/untouched) \end{verbatim}
\item \begin{verbatim} xcschememanagement.plist (Auto/untouched) \end{verbatim}
\item \begin{verbatim} project.pbxproj (Auto/untouched) \end{verbatim}
\item \begin{verbatim} Contents.json (Auto/untouched) \end{verbatim}
\item \begin{verbatim} Contents.json (Auto/untouched) \end{verbatim}
\item \begin{verbatim} Contents.json (Auto/untouched) \end{verbatim}
\item \begin{verbatim} Contents.json (Auto/untouched) \end{verbatim}
\item \begin{verbatim} BluetoothManager.swift (Christoph Schmid) \end{verbatim}
\item \begin{verbatim} ContentView.swift (Christoph Schmid) \end{verbatim}
\item \begin{verbatim} DeviceData.swift (Christoph Schmid) \end{verbatim}
\item \begin{verbatim} GasmeterApp.swift (Christoph Schmid) \end{verbatim}
\item \begin{verbatim} README.md (Christoph Schmid) \end{verbatim}
\item \begin{verbatim} .gitignore (Christoph Schmid) \end{verbatim}
\item \begin{verbatim} README.md (Christoph Schmid) \end{verbatim}
\end{itemize}
}
\subsubsection{author group}{
\subsubsection{Christoph Schmid}
\begin{itemize}
    \item \begin{verbatim}gasmeter_handheld_device.kicad \end{verbatim}
    \item \begin{verbatim}README.md \end{verbatim}
    \item \begin{verbatim} gasmeter_handheld_device.kicad_pcb  \end{verbatim}
    \item \begin{verbatim}gasmeter_handheld_device.kicad_pro \end{verbatim}
    \item \begin{verbatim}gasmeter_handheld_device.kicad_sch \end{verbatim}
    \item \begin{verbatim}gasmeter_handheld_device.kicad_pcb  \end{verbatim}
    \item \begin{verbatim}gasmeter_handheld_device.kicad_pro  \end{verbatim}
    \item \begin{verbatim}gasmeter_handheld_device.kicad_sch  \end{verbatim}
    \item \begin{verbatim}README.md  \end{verbatim}
    \item \begin{verbatim}BluetoothManager.swift \end{verbatim}
    \item \begin{verbatim}ContentView.swift  \end{verbatim}
    \item \begin{verbatim}DeviceData.swift \end{verbatim} 
    \item \begin{verbatim}GasmeterApp.swift \end{verbatim} 
    \item \begin{verbatim}README.md \end{verbatim}
    \item \begin{verbatim}.gitignore  \end{verbatim}
    \item \begin{verbatim} README.md \end{verbatim}
\end{itemize}

\subsubsection{Romeo Massoud}{
\begin{itemize}
    \item README.md
    graph.html
    index.html
    
\end{itemize}
}

\subsubsection{Maceo Pohl}{
\begin{itemize}
    \item README.md
    \item \begin{verbatim}README.md \end{verbatim}
    \item \begin{verbatim}handheld-gasmeter.ino \end{verbatim}
\end{itemize}
}

\subsubsection{Christoph Schmid, Maceo Pohl}{
\begin{itemize}
    \item lora-bluetooth-gateway.ino
\end{itemize}
}

\subsubsection{Christoph Schmid, Romeo Massoud}{
\begin{itemize}
    \item \begin{verbatim}__init__.py \end{verbatim}
    \item \begin{verbatim}__init__ app.py \end{verbatim}
    \item \begin{verbatim}__init__ README.md \end{verbatim}
    \item \begin{verbatim}__init__ requirements.txt \end{verbatim}
    \item \begin{verbatim}__init__ README.md \end{verbatim}
\end{itemize}
}

\subsubsection{Christoph Schmid, Romeo Massoud, Maceo Pohl}{
\begin{itemize}
    \item Smart gasmeter documentation
\end{itemize}
}

\begin{figure}[!htpb]
    \centering
    \includegraphics[width=\linewidth]{Figures/gasmeter_handheld_device.pdf}
    \caption[IVl Wireing Diagram.]{How the components of the wiring of the Handhelddevice }
    \label{fig:figure-02}
\end{figure}

\begin{figure}[!htpb]
    \centering
    \includegraphics[width=\linewidth]{Figures/WebFrontend1.jpg}
    \caption[Vl Webfrontend.]{Shows how the web fronted looks like. 4 different gases can be seen with there current ppm an an visual indicator if they are currently in an hazardous or harmless value range }
    \label{fig:WebFrontend1}
\end{figure}

\begin{figure}[!htpb]
    \centering
    \includegraphics[width=\linewidth]{Figures/Handhelddevice.jpg}
    \caption[VIl Handheld Device.]{Shows the Handheld Device in its enclosure }
    \label{fig:HandheldDevice}
\end{figure}

\subsection{Trello}

\begin{figure}[!htpb]
    \centering
    \includegraphics[width=\linewidth]{Figures/Trello1.jpg}
    \caption[TIl trello1.]{Shows the Trello Board at the time of the CONOPS MS Report}
    \label{fig:WebFrontend1}
\end{figure}

\begin{figure}[!htpb]
    \centering
    \includegraphics[width=\linewidth]{Figures/Trello2.jpg}
    \caption[TIIl trello2.]{Shows the Trello Board at the time of the Technology Demo MS Report}
    \label{fig:WebFrontend1}
\end{figure}

\begin{figure}[!htpb]
    \centering
    \includegraphics[width=\linewidth]{Figures/Trello3.png}
    \caption[TIIIl trello3.]{Shows the Trello Board at the time of the Architectural Spike MS Report }
    \label{fig:WebFrontend1}
\end{figure}

\begin{figure}[!htpb]
    \centering
    \includegraphics[width=\linewidth]{Figures/Trello4.jpg}
    \caption[TIVl trello4.]{Shows the Trello Board at the time of the Alpha Certification}
    \label{fig:WebFrontend1}
\end{figure}

\begin{figure}[!htpb]
    \centering
    \includegraphics[width=\linewidth]{Figures/Trello5.jpg}
    \caption[TVl trello5.]{Shows the Trello Board at the time of the Beta Certification}
    \label{fig:WebFrontend1}
\end{figure}

\begin{figure}[!htpb]
    \centering
    \includegraphics[width=\linewidth]{Figures/Trello6.png}
    \caption[TVIl trello6.]{Shows the Last Trello Board (at the time of the Certification ) }
    \label{fig:WebFrontend1}
\end{figure}



